%%%%%%%%%%%%%%%%%%%%%%%%%%%%%%%%%%%%%%%%%%%%%%%%%%%%%%%%%%%%%%%%%%%%%%
%     File: ExtendedAbstract_abstr.tex                               %
%     Tex Master: ExtendedAbstract.tex                               %
%                                                                    %
%     Author: Andre Calado Marta                                     %
%     Last modified : 2 Dez 2011                                     %
%%%%%%%%%%%%%%%%%%%%%%%%%%%%%%%%%%%%%%%%%%%%%%%%%%%%%%%%%%%%%%%%%%%%%%
% The abstract of should have less than 500 words.
% The keywords should be typed here (three to five keywords).
%%%%%%%%%%%%%%%%%%%%%%%%%%%%%%%%%%%%%%%%%%%%%%%%%%%%%%%%%%%%%%%%%%%%%%

%%
%% Abstract
%%
\begin{abstract}

\noindent
Flow cytometry plays a crucial role in biology and medicine by facilitating cell analysis. However, it faces significant challenges in terms of cost, integration, and portability. This work introduces a magnetic flow cytometric technique designed to address these limitations, specifically focusing on the detection and enumeration of circulating tumor cells in the bloodstream of cancer patients. The proposed system employs a microfluidic channel with integrated magnetic sensors to count tumor cells. Prior to introduction into the system, target tumor cells are bound with a magnetic cancer marker, enabling their magnetic detection within the microfluidic channel. The magnetic flow cytometer comprises three primary sections: analogue bias and amplification, analogue-to-digital conversion, and signal processing. This research centers on the analogue circuit responsible for interfacing with the magnetoresistive sensors. The primary objective of this interface is to overcome previous limitations and incorporate advanced features to facilitate comprehensive studies.
\\
%%
%% Keywords (max 5)
%%
\noindent{{\bf Keywords:}} Biomedical Analysis; Magnetic Flow Cytometry; Magnetoresistive Sensors; Ultra-Low Noise Platform; Circuit Design \\

\end{abstract}

