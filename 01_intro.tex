\section{Introduction}
\label{sec:intro}

This document highlights a work conducted at Instituto de Engenharia de Sistemas e Computadores (INESC), focusing on developing a Magnetic Flow Cytometer (MFC) system for early cancer detection \cite{JoseC_thesis, PMID24761029}. The objective is to enhance the analog front-end of the MFC, addressing its limitations and incorporating novel features to facilitate advanced studies.

The primary aim of this work is to achieve a high Signal-to-Noise Ratio (SNR) in the system by carefully managing noise sources and implementing effective noise reduction strategies. This involves optimizing the analog interface, which encompasses the electronic circuits responsible for biasing the sensors, establishing optimal electrical conditions for signal generation, and amplifying and filtering the signal prior to digitization. The platform can provide improved sensitivity and accuracy for magnetic field analysis in the cytometer system by minimising noise and ensuring a favourable SNR.

Additionally, the new platform has increased analog channels in the system, enhancing its capability for detecting circulating tumor cells responsible for tumor metastasis. The integration of microfluidics, Magnetoresistive (MR) sensors, and microelectronics allows for the precise detection and enumeration of these cells in the bloodstream of cancer patients.

The circuits in this work were designed by myself and research teams from INESC. Teams at INESC and International Iberian Nanotechnology Laboratory developed the magnetic sensors used in the cytometer system. The collaboration and expertise of everyone have resulted in robust and cutting-edge circuitry and sensor technology, ensuring a state-of-the-art platform with reliable and accurate magnetic field analysis capabilities in the MFC system.