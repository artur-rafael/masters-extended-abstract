% A Theory section should extend, not repeat, the background to the
% article already dealt with in the Introduction and lay the
% foundation for further work.

\section{Magnetic Flow Cytometry}
\label{sec:mfc}

The MFC is a powerful tool used for characterizing immune cell phenotypes and monitoring various diseases, including solid tumors and hematological malignancies \cite{cpim.40}. In this project, MR sensors are used paired with a permanent magnet in the MFC system. To detect non-magnetic cells, magnetic markers, known as Magnetic Nano Particles (MNPs), are bonded to the analytes, generating a magnetic field near the permanent magnet.

Efficient labeling is crucial for the MFC detection system, as the signal from the MR sensors depends on the amount of MNPs on the cell's surface. The MFC system has a dynamic range, and the MR sensor response is proportional to the average fringe field generated by the MNPs. Therefore, achieving a high labeling efficiency is essential for accurate quantitative analysis.

To ensure effective labeling, specific ligand molecules \cite{Freitas2017SpintronicBF} are used to create a bridge between the magnetic particles and the cell's outer surface. The bonding process efficiency varies depending on the size of the magnetic label. Researchers have made significant efforts to choose or create the correct bead and antibody conjugates to optimize the labeling process.

Once the cells are labeled with MNPs, they are magnetized and injected into a microfluidic channel within the system. The detection mechanism in the MFC platform relies on the interaction between the tagged analytes and the MR sensors. The permanent magnet magnetizes the MNPs, attracting them closer to the sensor surface through vertical magnetophoresis. The MR sensors pick up the magnetic field \cite{DiogoC_thesis} produced by the particles, resulting in a change in resistance according to Equation \ref{eq:mr-change}.
\begin{equation}
\label{eq:mr-change}
    R_{MR}(H) = R_{NOM} + \Delta R(H) \quad [\Omega]
\end{equation}

The resistance of the MR sensors varies depending on the orientation and intensity of the magnetic field. These sensors have a fixed nominal resistance ($R_{NOM}$) and a field-dependent resistance variation ($\Delta R(H)$). The interaction between the sensors and particles produces a monocycle pulse if the excitation field is perpendicular to the particle, resulting in a specific signal pattern.