\section{Conclusions}
\label{sec:concl}

The research conducted in this dissertation focuses on the development of a discrete electronics system for interfacing MR sensors, with a specific emphasis on meeting the requirements for early cancer diagnosis applications. To address the challenges associated with noise propagation and system degradation, an ultra-low noise interface was developed. The architecture underwent significant changes, including modifications to the biasing topology and the incorporation of a flexible reference voltage control. The effectiveness of different sensor placement within the feedback loop was explored, providing alternative configurations for noise reduction. Additionally, improvements were made to the amplification scheme and saturation detection circuitry. The integration of a new circuit allowed for enhanced sensor selection and utilization. Compatibility with the larger MFC system was achieved through supplementary circuits and the implementation of gain reduction multiplexing and level shifters. The power supply configuration was redesigned to meet increased power demands and offer expanded power source options. These advancements lay the groundwork for further progress in MR sensor interfacing and contribute to the development of a robust platform for early cancer diagnosis and related applications.